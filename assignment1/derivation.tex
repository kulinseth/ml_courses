\title{Assignment 1: Written Exercises}
\author{Kulin Seth}
\date{\today}
\documentclass[12pt]{article}

\usepackage[margin=1in]{geometry}  % set the margins to 1in on all sides
\usepackage{graphicx}              % to include figures
\usepackage{amsmath}               % great math stuff
\usepackage{amsfonts}              % for blackboard bold, etc
\usepackage{amsthm}                % better theorem environments
\newcommand{\E}{\mathrm{E}}
\newcommand{\Var}{\mathrm{Var}}
\newcommand{\Cov}{\mathrm{Cov}}

% For derivatives
\newcommand{\deriv}[2]{\frac{\mathrm{d}}{\mathrm{d} #1}{#2}}

% For partial derivatives
\newcommand{\pderiv}[2]{\frac{\partial}{\partial #1} (#2)}

% Integral dx
\newcommand{\dx}{\mathrm{d}x}

\begin{document}
\maketitle
\pagebreak

\section{Question 1}
Variance of a sum. Show that the variance of a sum  is

$\Var[X + Y] = \Var[X] + \Var[Y ] + 2 * \Cov[X,Y]$

where $\Cov[X,Y ]$ is the covariance between random variables X and Y.


\section {Question 2}

\pagebreak
\section {Question 3}
Gradient and Hessian of log-likelihood for logistic regression.
\subsection{a} 
Let

\begin{equation}
    \rho(a) = \frac{1}{1+\exp^{-a}}
\end{equation}

be the sigmoid function. Show that :

\begin{equation}
    \deriv{a}{\rho(a)} = \rho(a)(1 − \rho(a))
\end{equation}


\textbf{Proof}

Differentiating $\rho(a)$ w.r.t $a$ we get :

\textbf{L.H.S}

    \begin{align*}
        \deriv{a}{\rho(a)}  &= \deriv{a}{\left( \frac{1}{1 + e^{-a}} \right) }\\
                            &= \left( \frac{-1}{(1 + e^{-a})^{2}} \right) \deriv{a}{\left( 1 + e^{-a} \right)} 
                            && \text{Reciprocal rule:}\frac{1}{f} \rightarrow \frac{-f\prime}{f^2} \\
                            &= \left( \frac{-1}{(1 + e^{-a})^2} \right) \left( -{e^{-a}} \right)
                            && \text{Derivative of:} e^{-a} \rightarrow -e^{-a} \\
                            &= \frac{e^{-a}}{(1 + e^{-a}) ^ 2} && \text {Simplifying} 
    \end{align*}

Substituting $\rho(a)$ on the R.H.S we get:

\textbf{R.H.S}

        \begin{align*}
            \rho(a)(1 - \rho(a)) &= \frac{1}{1+e^{-a}} \left( 1-\frac{1}{1+e^{-a}} \right)
                                 && \text{(substituting)}\\
                                 &= \frac{\exp^{-a}}{(1 + \exp^{-a})^2} && \text{(simplifying)}
        \end{align*}

$R.H.S \iff L.H.S$ hence proved.

\pagebreak

\subsection{b} 

Using the previous result and the chain rule of calculus, derive the expression for the gradient
of the log likelihood given in HTF Eqn. 4.21.

Equation (4.20) states:

\begin{equation} \label{eq:4.20a}
l(\beta) = \displaystyle\sum_{i=1}^{N} \left \{ y_i \log p(x_i;\beta)
+ (1-y_i) \log (1 - p(x_i;\beta)) \right \}
\end{equation}

\begin{equation} \label{eq:4.20b}
l(\beta) = \displaystyle\sum_{i=1}^{N} \left \{ y_i \beta^T x_i 
- log (1 + \exp^{\beta^{T}x_i}) \right\}
\end{equation}

We can get from \eqref{eq:4.20a} to \eqref{eq:4.20b} by substituting 

    \[
        p(x_i;\beta) = \frac{e^{{\beta}^Tx_i}}{1+{e^{{\beta}^Tx_i}}}
    \]

\textbf{PROOF}

    To Prove:

    \[
        \pderiv{\beta}{l(\beta)} = \displaystyle\sum_{i=1}^{N} x_i(y_i-p(x_i;\beta))
    \]

    Differentiating \eqref{eq:4.20b} w.r.t to $\beta$ on both sides we get:
    
    \begin{align*}
        \pderiv{\beta}{l(\beta)}&= \pderiv{\beta}{
                                \displaystyle\sum_{i=1}^{N}(y_i\beta^Tx_i -log(1 + \exp^{\beta^{T}x_i}))}\\
                                &=\displaystyle\sum_{i=1}^{N} (\pderiv{\beta}{y_i\beta^Tx_i}) -
                                (\pderiv{\beta}{log(1 + \exp^{\beta^{T}x_i}))} && \text{moving derivative inside}\\
                                &=\displaystyle\sum_{i=1}^{N}y_ix_i -\frac{1}{1 + e^{\beta^{T}x_i}}(x_i e^{\beta^{T}x_i})
    \end{align*}

\subsection{c} 

As noted in HTF Eqn. 4.25, the Hessian matrix for the log likelihood can be written 
(up to a sign) as $X^TWX$ . Prove that this matrix is positive definite.

\end{document}
