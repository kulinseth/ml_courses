\documentclass[11pt]{article}
\usepackage[latin1]{inputenc}
\usepackage{amsmath}
\usepackage{epsfig}
\usepackage{pgf}
\usepackage{graphicx}
 \usepackage{amssymb}
\usepackage{url}
\def\urltilda{\kern -.15em\lower .7ex\hbox{\~{}}\kern .04em}

    \oddsidemargin  0.0in
    \evensidemargin 0.0in
    \textwidth      6.5in
    \headheight     0.0in
        \headsep        0.0in
    \topmargin      0.0in
    \textheight=9.0in

\begin{document}
\title{Machine Learning 6.867: HW 1}
\author{Anonymous Authors}
\maketitle



\section{Probabilities [30 pt, Field Cady]}
Probability is, in many ways, the most fundamental mathematical technique for
machine learning. This problem will review several basic notions from
probability and make sure that you remember how to do some elementary proofs.

Recall that for a discrete random variable (r.v.) X whose values are integers,
we frequently use the notation $P(X=x)$ for the probability its value is $x$. If the r.v. Y is \emph{continuous}, we typically use a ``density function'' $p(Y=y)$. The
conditions for $P(X=x)$ to be a valid probability distribution are that
$\sum_{-\infty}^{\infty}p(X=x)=1$ and $P(X=x) \geq 0 \forall x$. Similarly for $p(Y=y)$ to
be a valid continuous distribution, $\int_{-\infty}^{\infty}p(Y=y)dy=1$ and $p(Y=y)
\geq 0$.

Sometimes the underlying probability space has more than one variable (for
example, the height and weight of a person).  In this case, we may use notation
like $p(X=x,Y=y)$ to denote the probability density function in several
dimensions.

\subsection{Expectations [10 pt]} Expectation is another word for ``mean''.  They are
similar to ``average''; the difference is that an average usually refers to the
average of some data we collected, whereas expectation usually refers to the
underlying distribution from which we have sampled.  For a discrete r.v. X, the
expectation value is defined to be $E[X]=\sum_{-\infty}^{\infty}i P(X=i)$. If Y
is a continuous random variable, $E[Y]=\int_{-\infty}^{\infty}y p(y)dy$.
\begin{enumerate}
\item Show that, for discrete r.v. W and Z, $E[W+Z]=E[W]+E[Z]$.
\item Show that, for \emph{continuous} r.v. W and Z, $E[W+Z]=E[W]+E[Z]$.
\end{enumerate}

\subsection{Independence [10 pt]}
Intuitively, two r.v. X and Y are ``independent'' if knowledge of the value of
one tells you nothing at all about the value of the other. Precisely, if X and
Y are discrete, independence means that $P(X=x,Y=y)=P(X=x)P(Y=y)$, and if they are continuous,
$p(X=x,Y=y)=p(X=x)p(Y=y)$.  Show the following, for \emph{independent} r.v. X
and Y:
\begin{enumerate}
\item If X and Y are discrete, $E[XY]=E[X]E[Y]$.
\item If X and Y are \emph{continuous}, $E[XY]=E[X]E[Y]$.
\end{enumerate}

\subsection{Variance [10 pt]}
Variance for a r.v. X indicates how ``spread out'' the distribution is.
Precisely, if $\bar{X}=E[X]$, the variance is defined to be
$Var[X]=E[(X-\bar{X})^2]$. Show the following:
\begin{enumerate}
\item For a (discrete \emph{or} continuous) random variable X,
    $Var[X]=E[X^2]-(E[X])^2$.

    \emph{Hint: you don't have to treat
    the discrete and continuous cases separately; it can be done
    just using expectation.}
\item Let X be continuous, and let it follow the celebrated
    normal distribution: $p(X=x)=\frac{1}{\sqrt{2\pi \sigma^2}}e^{(x-\mu)^2/(2\sigma^2)}$,
    where $\sigma$ can be any positive real number, and $\mu$ can be
    any real number.  Show that $Var[X]=\sigma^2$.
\end{enumerate}


\section{Decision Trees and Information Theory [35+5 pt, Ni Lao]}
Suppose a discrete variable $X$ has $n$ categories $1...n$. It's entropy is defined as \begin{displaymath}
H(X)=-\sum_i {P(X=i)\log{P(X=i)}}.
\end{displaymath}
Suppose another variable $Y$ has distribution $P(Y=j)$, and
joint distribution $P(X=i,Y=j)$. Then their mutual information is
\begin{displaymath}
I(X;Y)=\sum_{i,j} P(X=i,Y=j) \log \frac{P(X=i,Y=j)}{P(X=i) P(Y=j)}=H(X)-H(X|Y)=H(Y)-H(Y|X)
\end{displaymath}
A \textit{K-nary code} $C$ for $X$ is a mapping from $X$ to a string
(code word) $C(X)$ of which each character can have $K$ values. A
\textit{prefix code} is a code for which no code word is a prefix of
other code word. The average code length is defined as $L(C)=\sum_i {P(X=i) l_i}$
where $l_i$ is the length of $C(i)$. It can be proved that $H(X)$ is
the minimum average code length needed to encode $X$, and the
minimum is reached if and only if $P(X=i)= K^{-l_i}$.

%to 0, then we have $q_i=p_i$
\begin{enumerate}
%\item
%\textbf{The optimal encoding problem [10 pt]} A (binary) \textit{code} $C$ for
%$X$ is a mapping from $X$ to a binary string (code word) $C(X)$. A
%\textit{prefix code} is a code for which no code word is a prefix of other code
%word. The average code length is $L(C)=\sum_i {p_i l_i}$ where $l_i$ is the
%length of $C(i)$. \textbf{Proof} that $H(X)$ is the minimum average code length
%needed to encode $X$.
%%For simplicity, let's assume that $p_i=2^{-m_i}$, where $n_i$ is an integer.
%\textbf{Hint:} Minimizes $L(C)$ with the constraint that $\sum_i {2^{-l_i}}
%\leq 1$ (Kraft Inequality). \textbf{More hint:} Use the Lagrangian Multiplier
%Method.

\item
\textbf{The 9 ball problem [10 pt]} There are 9 metal balls. One of then is
heavier than the others. Please design a strategy given a scale to find out
which one is heavier with the least number of expected trials. Then show that
it is optimal from information point of view. \textbf{Hint:} connect the
average code length with the expected number of tests.

\item
\textbf{Information gain and optimal encoding [10 pt]} show that, when using
$K$-outcome tests, if each test $Y_t$ has optimal mutual information
$I(Y_t;X|Y_1 ...Y_{t-1})=1$ (in base $K$ numeral system) conditioned on all
previous tests, then the expected number of tests is optimal. Assume that the
mapping from $X$ to $Y_t$ are deterministic ($H(Y_t|X)=0$). \textbf{Hint:}
$H(Y) \geq I(X;Y)$.
%\textbf{More hint:} use the result of the first problem.
%the expected code length  a $k$-nary code  encoding $
%\textbf{More hint:} $H(X|Y) \geq H(X)-H(Y)$

\item
\textbf{The ID3 algorithm [10 pt]} If we run the ID3 algorithm on the 9 metal
ball problem, will it generate the optimal decision tree? \textbf{Hint:} treat
the outcome of comparing any two sets of balls as a feature, and compare their
information gains.

%\item
%\textbf{The syphilis test problem [5 pt]} Now we have blood sample of $n$
%person ($n \gg 1$), and we know that $m$ ($m \ll n$)  of them have been
%infected by syphilis. In order to find out which ones are they, what is the
%minimum expected number of syphilis test we need to do? Briefly describe your
%strategy. \textbf{Hint:} we can do syphilis test on a mixture of several
%people's blood.
%%each of them have probability $p$ ($p \ll 1$) been infected by syphilis

\item
\textbf{The number guessing problem [5 pt]} Alice has a favorite four digit number
 (e.g. $0123$), and she wants Bob to guess it. At each
iteration, Bob say a four digit number (e.g. $3210$), and then Alice tell him
how many digit is correct (four in this example), and how many of them have
correct position (zero in this example). From information point of view, what is the minimum expected number of guesses
Bob has to make in order to identify the number?

\item
\textbf{[Extra 5 pt]}
%Briefly describe your strategy.
Try the number guessing game a few times with your friend. Did you achieve your estimated lower bound? Why?

\end{enumerate}
%%%%%%%%%%%%%%%%%%%%%%%%%%%%%%%%%%%%%%%%%%%%%%%%%%%%%%%%%%%%
\section{KNN and Decision Trees [35 pt, Amr]}

\begin{enumerate}
\item \textbf{(10 points)}
For each of the following figures, we are given a few data points in
2-d space,each of which is labeled as either positive (blue) or
negative (red). Assuming that we are using L2 distance as a distance
metric, draw the decision boundary for 1-NN for each case. For
example, the decision boundary for the dataset in 1.b is given in
figure 2.a.

\begin{figure*}[h!]
\centering
  \begin{tabular}{cccc}
  \end{tabular}
  \caption{Data sets for problem 3.}
 \end{figure*}


\item \textbf{(3 points)} In class we have mentioned that NN is a \emph{lazy}
classifier that needs to store all training instances until test
time. However,  in this problem we were able to draw a decision
boundary for the 1-NN classifier. If we decided to store this
decision boundary instead of storing all training data, would that
\emph{always} result in an improvement in terms of storage  (memory)
requirement for this classifier?  [Please answer in no longer than
2-5 sentences]

\item \textbf{(2 point)}
Decision trees are known as batch learners that require the
availability of all training data to build the tree. Thus the
arrival of additional training data needs to be handled carefully.
Does KNN suffer from this problem and why?


\item
We would like to build a decision tree over the dataset in Figure
1.d. Each data item is described using two continuous attributes
$(x,y)$. One way of handing continuous attributes is discretization,
here we will use binary-discretization and test if a given attribute
is $\geq$ a given threshold which we will take to be the midpoint.
For instance, in Figure 1.d, the range of both $X$ and $Y$ is
[-4,4], thus the test at the first level of the tree is either $x
\geq 0$ or $y \geq 0$. In the next level, we might test for $x \geq
2$,$x \geq -2$, $y \geq 2$, etc.  In other words, we  always
consider the current range of both attributes in the training data
at a given leaf node, and formulate a test that would divide this
range into two equal parts.  For instance, if we apply this scheme
to the data in Figure 1.c, we get the decision boundary in Figure
2.b (other solutions are possible as well).

\begin{enumerate}
\item \textbf{(2 points)}
What would you expect the training error to be if we apply the above scheme
to the data in Figure 1.d? and why?

\item \textbf{(7 points)}
To limit the size of the resulting decision tree,
we will stipulate that any single attribute is
tested at most twice on any path from the root of the tree to a a
leaf node. With this restriction,  draw the decision boundary of a
possible decision tree over the data in Figure 1.d. you don't need
to do any calculations, just draw the decision boundary that
corresponds to a suitable decision tree.
\end{enumerate}


\item \textbf{(4 points)} Using the intuition you gained in this
problem, state one advantage of Decision Trees over KNN and one advantage of
KNN over Decision Trees.
\end{enumerate}

\end{document}
