\title{Chapter1 : Introduction}
\author{Kulin Seth}
\date{\today}
\documentclass[12pt]{article}
\usepackage{hyperref}
\usepackage[margin=1in]{geometry}  % set the margins to 1in on all sides
\usepackage{graphicx}              % to include figures
\usepackage{amsmath}               % great math stuff
\usepackage{amsfonts}              % for blackboard bold, etc
\usepackage{amsthm}                % better theorem environments
\newcommand{\E}{\mathrm{E}}
\newcommand{\Var}{\mathrm{Var}}
\newcommand{\Cov}{\mathrm{Cov}}

\begin{document}
The book introduces Statistical Learning by providing three different examples.

\section{Email Spam}
The data used was downloaded from \href{https://archive.ics.uci.edu/ml/datasets/Spambase}{Spam}
dataset as part of UCI reporsitory. Its a classification problem where you want to filter out
spam without losing any relevant emails. The data has features like certain \textit{WORD} 
frequency, \textit{Special} character frequency, word length etc. This is used to classify
whether email is spam or not.

\section{Prostate Cancer}
The goal is to predict the \textit{lpsa} log of PSA antigen in prostate cacncer using different
clinical features. This was more of a regression problem.

\section{Digit Recognition}
Handwritten ZIP code database is used to classify different digits for USPS. The images have
been normalized to 16x16 image. This is a classification problem. Here the error-rate needs
to be low, so above a certain threshold the mail get classified into "Dont Know" category.
These are then classified by hand.

\section{Gene Expression Microarrays}

\end{document}
